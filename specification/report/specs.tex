%%%%%%%%%%%%%%%%%%%%%%%%%%%%%%%%%%%%%%%%%
% University Assignment Title Page 
% LaTeX Template
% Version 1.0 (27/12/12)
%
% This template has been downloaded from:
% http://www.LaTeXTemplates.com
%
% Original author:
% WikiBooks (http://en.wikibooks.org/wiki/LaTeX/Title_Creation)
%
% License:
% CC BY-NC-SA 3.0 (http://creativecommons.org/licenses/by-nc-sa/3.0/)
% 
% Instructions for using this template:
% This title page is capable of being compiled as is. This is not useful for 
% including it in another document. To do this, you have two options: 
%
% 1) Copy/paste everything between \begin{document} and \end{document} 
% starting at \begin{titlepage} and paste this into another LaTeX file where you 
% want your title page.
% OR
% 2) Remove everything outside the \begin{titlepage} and \end{titlepage} and 
% move this file to the same directory as the LaTeX file you wish to add it to. 
% Then add \input{./title_page_1.tex} to your LaTeX file where you want your
% title page.
%
%%%%%%%%%%%%%%%%%%%%%%%%%%%%%%%%%%%%%%%%%

%----------------------------------------------------------------------------------------
%	PACKAGES AND OTHER DOCUMENT CONFIGURATIONS
%----------------------------------------------------------------------------------------

\documentclass[12pt]{article}

\usepackage[utf8]{inputenc}
\usepackage[T1]{fontenc}
\usepackage{lmodern}
\usepackage{parselines}
\usepackage[portuguese]{babel}
\usepackage{graphicx}
\usepackage[document]{ragged2e}
\usepackage{listings}
\usepackage{xcolor}
\usepackage[margin=0.8in]{geometry}		
\usepackage{amsmath}
\usepackage{hyperref}
\usepackage{url}
\usepackage{float}

\graphicspath{ {images/} }

\colorlet{punct}{red!60!black}
\definecolor{background}{HTML}{EEEEEE}
\definecolor{delim}{RGB}{20,105,176}
\colorlet{numb}{magenta!60!black}


\lstdefinelanguage{json}{
    basicstyle=\normalfont\ttfamily,
    numbers=left,
    numberstyle=\scriptsize,
    stepnumber=1,
    numbersep=8pt,
    showstringspaces=false,
    breaklines=true,
    frame=lines,
    backgroundcolor=\color{background},
    literate=
     *{0}{{{\color{numb}0}}}{1}
      {1}{{{\color{numb}1}}}{1}
      {2}{{{\color{numb}2}}}{1}
      {3}{{{\color{numb}3}}}{1}
      {4}{{{\color{numb}4}}}{1}
      {5}{{{\color{numb}5}}}{1}
      {6}{{{\color{numb}6}}}{1}
      {7}{{{\color{numb}7}}}{1}
      {8}{{{\color{numb}8}}}{1}
      {9}{{{\color{numb}9}}}{1}
      {:}{{{\color{punct}{:}}}}{1}
      {,}{{{\color{punct}{,}}}}{1}
      {\{}{{{\color{delim}{\{}}}}{1}
      {\}}{{{\color{delim}{\}}}}}{1}
      {[}{{{\color{delim}{[}}}}{1}
      {]}{{{\color{delim}{]}}}}{1},
}

\definecolor{dkgreen}{rgb}{0,0.6,0}
\definecolor{gray}{rgb}{0.5,0.5,0.5}
\definecolor{mauve}{rgb}{0.58,0,0.82}

\lstset{frame=tb,
  language=Java,
  aboveskip=3mm,
  belowskip=3mm,
  showstringspaces=false,
  columns=flexible,
  basicstyle={\small\ttfamily},
  numbers=none,
  numberstyle=\tiny\color{gray},
  keywordstyle=\color{blue},
  commentstyle=\color{dkgreen},
  stringstyle=\color{mauve},
  breaklines=true,
  breakatwhitespace=true,
  tabsize=3
}

\hypersetup{
  colorlinks, linkcolor=black
}

\begin{document}

\begin{titlepage}

\newcommand{\HRule}{\rule{\linewidth}{1mm}} % Defines a new command for the horizontal lines, change thickness here

\center % Center everything on the page
 
%----------------------------------------------------------------------------------------
%	HEADING SECTIONS
%----------------------------------------------------------------------------------------

\includegraphics{feup.jpg}

\textsc{\large Sistemas de Informação}\\[0.8cm] % Major heading such as course name
\textsc{\large 4º ano do Mestrado Integrado em Engenharia Informática e Computação}\\[0.8cm] % Minor heading such as course title

%----------------------------------------------------------------------------------------
%	TITLE SECTION
%----------------------------------------------------------------------------------------

\HRule \\[1.2cm]
{ \huge \bfseries \textit{Sales Representative Automation}}\\[0.6cm] % Title of your document
{ \large \bfseries Especificação Funcional} \\[0.6cm]
\HRule \\[2cm]
 
%----------------------------------------------------------------------------------------
%	AUTHOR SECTION
%----------------------------------------------------------------------------------------


% If you don't want a supervisor, uncomment the two lines below and remove the section above
\Large \emph{Authors:}\\[0.5cm] \normalsize
Filipa \textsc{Ramos}\\[0.1cm] 
- up201305378@fe.up.pt\\[0.1cm]  
Gil \textsc{Domingues}\\[0.1cm]  
- up201304646@fe.up.pt\\[0.1cm]
Pedro \textsc{Pontes}\\[0.1cm]
- up201305367@fe.up.pt\\[0.1cm] % Your name
Pedro \textsc{Melo}\\[0.1cm]
- up201305618@fe.up.pt\\[2cm] % Your name

%----------------------------------------------------------------------------------------
%	DATE SECTION
%----------------------------------------------------------------------------------------

{\large \today}\\[0cm] % Date, change the \today to a set date if you want to be precise

%----------------------------------------------------------------------------------------
%	TABLE OF CONTENTS & LISTS OF FIGURES AND TABLES
%----------------------------------------------------------------------------------------

\tableofcontents

%----------------------------------------------------------------------------------------
%	INTRODUÇÃO
%----------------------------------------------------------------------------------------

\section{Introdução} 

\justify\normalsize
Muito tem sido dito acerca do papel da tecnologia no marketing e como esta pode substituir os processos de venda tradicionais. No entanto, os representantes de vendas ainda são fundamentais no fecho de negócios: falando com clientes por telefone ou pessoalmente, são eles quem «vende» aos clientes os méritos e benefícios de um dado produto, levando à geração de lucros. A tecnologia não consegue substituir, por completo, a interação humana. 

Com efeito, a equipa de vendas continua a ser a face de uma empresa – ainda que nuns setores mais que noutros. Porque essas empresas despendem uma considerável quantidade de tempo e dinheiro com as equipas de vendas, importa geri-las de forma tão eficiente e eficaz quanto possível. 

Para tal, existem soluções SFA – Sales Force Automation –, software que permite automatizar tarefas como controlo de inventário, processo de vendas, e registo de interações com clientes. São normalmente integradas na componente de Costume Relationship Management (CRM) de um sistema de Enterprise Resource Planning (ERP), existindo várias alternativas no mercado: SAP CRM (Sales), Microsoft Dynamics CRM e Oracle Siebel são alguns exemplos. 

Realizado no âmbito da unidade curricular de Sistemas de Informação, o presente documento detalha a especificação de uma aplicação web que oferecerá as funcionalidades de uma solução SFA. Ao longo do documento, é feita uma descrição do projeto, enumerando-se as funcionalidades a implementar, e identificam-se os pontos de ligação com o ERP Primavera. 

%----------------------------------------------------------------------------------------

\newpage % Start the article content on the second page, remove this if you have a longer abstract that goes onto the second page

%----------------------------------------------------------------------------------------
%	ESPECIFICAÇÃO
%----------------------------------------------------------------------------------------

\section{Especificação}

\subsection{Descrição}
\justify\normalsize
O projeto consiste no desenvolvimento de uma aplicação web que permitirá aos re-presentantes de vendas de uma empresa organizar da sua agenda, mantendo o registo de contactos com clientes – incluindo sumários de visitas a clientes –, registar ordens de venda, gerir o perfil de clientes.

Existirá uma camada de autenticação, através da qual os representantes de vendas poderão registar-se e autenticar-se,

Gestão de potenciais clientes

Gestão de catálogos de produtos

Com este trabalho pretende-se, igualmente, a aquisição de conhecimentos no que respeita ao uso de um ERP, no caso, o Primavera, designadamente, no que respeita à definição de Master Data – no caso, produtos, clientes, vendas.


\subsection{Funcionalidades}


\section{Core Views}


\section{Interoperabilidade}

\subsection{Clientes}

\begin{table}[H]
\centering
\caption{Tabela do método get\_client.}
\label{my-label}
\begin{tabular}{|c|l|}
\hline 
\textbf{Webservice ID}                     & \multicolumn{1}{c|}{get\_client}                                                                                                                                                                                        \\ \hline
\multicolumn{1}{|l|}{\textbf{Description}} & Retorna a informação de um cliente específico.                                                                                                                                                                                    \\ \hline
\textbf{Core Views}                        & CLI\_PAGE                                                                                                                                                                                                                          \\ \hline
\textbf{Path}                              & \url{http://sinf-scheduler.com/client=<clientID>}                                                                                                                                                          \\ \hline
\textbf{Verb}                              & \multicolumn{1}{c|}{GET}                                                                                                                                                                                                \\ \hline
\textbf{Input}                             & \multicolumn{1}{c|}{clientID}                                                                                                                                                                                           \\ \hline
\textbf{Output}                            & \begin{tabular}[c]{@{}l@{}}\{\\      "id": "EFACSA",\\      "nome": "EFACEC",\\      "nome\_fiscal": "EFACEC SA",\\ \\      "morada": "Porto",\\      "telefone": 969509655,\\      "contribuinte": 989922455\\ \}\end{tabular} \\ \hline
\end{tabular}
\end{table}

\begin{table}[H]
\centering
\caption{Tabela do método get\_all\_clients.}
\label{my-label}
\begin{tabular}{|c|l|}
\hline 
\textbf{Webservice ID}                     & \multicolumn{1}{c|}{get\_all\_clients}                                                                                                                                                                                        \\ \hline
\multicolumn{1}{|l|}{\textbf{Description}} & Retorna todos os clientes.                                                                                                                                                                                    \\ \hline
\textbf{Core Views}                        & HOM\_PAGE                                                                                                                                                                                                                          \\ \hline
\textbf{Path}                              & \url{http://sinf-scheduler.com/homepage}                                                                                                                                                          \\ \hline
\textbf{Verb}                              & \multicolumn{1}{c|}{GET}                                                                                                                                                                                                \\ \hline
\textbf{Input}                             & \multicolumn{1}{c|}{userID}                                                                                                                                                                                           \\ \hline
\textbf{Output}                            & \begin{tabular}[c]{@{}l@{}}\{  "Clientes": [ \\ \{ "id": "EFACSA",\\     "nome": "EFACEC",\\     "nome\_fiscal": "EFACEC SA",\\ \\      "morada": "Porto",\\      "telefone": 969509655,\\      "contribuinte": 989922455\\ \}, \\ \{ \\ "id":  "LIMA", \\ "nome":  "Empreendimentos do Lima", \\ "nome\_fiscal": "Empreendimentos do Lima, Lda", \\ "morada": "Bairro PSP", \\ "telefone": , \\ "contribuinte": 202075133 \\ \}  \\ ]\}
\end{tabular} \\ \hline
\end{tabular}
\end{table}


\begin{table}[H]
\centering
\caption{Tabela do método add\_client.}
\label{my-label}
\begin{tabular}{|c|l|}
\hline 
\textbf{Webservice ID}                     & \multicolumn{1}{c|}{add\_client}                                                                                                                                                                                        \\ \hline
\multicolumn{1}{|l|}{\textbf{Description}} & Adiciona um cliente.                                                                                                                                                                                    \\ \hline
\textbf{Core Views}                        & NEW\_CLI                                                                                                                                                                                                                         \\ \hline
\textbf{Path}                              & \url{http://sinf-scheduler.com/new_client}                                                                                                                                                          \\ \hline
\textbf{Verb}                              & \multicolumn{1}{c|}{POST}                                                                                                                                                                                                \\ \hline
\textbf{Input}                             & \multicolumn{1}{c|}{nome, nome\_fiscal, email, morada, telefone, contribuinte}                                                                                                                                                                                           \\ \hline
\textbf{Output}                            & \begin{tabular}[c]{@{}l@{}}\{\\  "resultado": "OK" \\ \}\end{tabular} \\ \hline
\end{tabular}
\end{table}

\subsection{Reuniões}

\begin{table}[H]
\centering
\caption{Tabela do método add\_meeting.}
\label{my-label}
\begin{tabular}{|c|l|}
\hline 
\textbf{Webservice ID}                     & \multicolumn{1}{c|}{add\_meeting}                                                                                                                                                                                        \\ \hline
\multicolumn{1}{|l|}{\textbf{Description}} & Adiciona uma reunião.                                                                                                                                                                                    \\ \hline
\textbf{Core Views}                        & NEW\_MEE
\\ \hline
\textbf{Path}                              & \url{http://sinf-scheduler.com/new_meeting}                                                                                                                                                          \\ \hline
\textbf{Verb}                              & \multicolumn{1}{c|}{POST}                                                                                                                                                                                                \\ \hline
\textbf{Input}                             & \multicolumn{1}{c|}{descricao, data, tempo, notas}                                                                                                                                                                                           \\ \hline
\textbf{Output}                            & \begin{tabular}[c]{@{}l@{}}\{\\  "resultado": "OK" \\ \}\end{tabular} \\ \hline
\end{tabular}
\end{table}

\begin{table}[H]
\centering
\caption{Tabela do método edit\_meeting.}
\label{my-label}
\begin{tabular}{|c|l|}
\hline 
\textbf{Webservice ID}                     & \multicolumn{1}{c|}{edit\_meeting}                                                                                                                                                                                        \\ \hline
\multicolumn{1}{|l|}{\textbf{Description}} & Edita uma reunião.                                                                                                                                                                                    \\ \hline
\textbf{Core Views}                        & EDIT\_MEE????????????
\\ \hline
\textbf{Path}                              & \url{http://sinf-scheduler.com/edit_meeting}                                                                                                                                                          \\ \hline
\textbf{Verb}                              & \multicolumn{1}{c|}{POST}                                                                                                                                                                                                \\ \hline
\textbf{Input}                             & \multicolumn{1}{c|}{descricao, data, tempo, notas}                                                                                                                                                                                           \\ \hline
\textbf{Output}                            & \begin{tabular}[c]{@{}l@{}}\{\\  "resultado": "OK" \\ \}\end{tabular} \\ \hline
\end{tabular}
\end{table}

\subsection{Armazéns}

\begin{table}[H]
\centering
\caption{Tabela do método get\_warehouse.}
\label{my-label}
\begin{tabular}{|c|l|}
\hline
\textbf{Webservice ID}                     & \multicolumn{1}{c|}{get\_warehouse}                                                                                                                                                                                        \\ \hline
\multicolumn{1}{|l|}{\textbf{Description}} & Retorna a informação dos armazens.
 \\ \hline
\textbf{Core Views}                        & \multicolumn{1}{c|}{PRO\_PAGE}                                                                                                                                                                                                                        \\ \hline
\textbf{Path}                              & \url{http://<domain>/api/warehouse=<warehouseID>}                                                                                                                                                          \\ \hline
\textbf{Verb}                              & \multicolumn{1}{c|}{GET}                                                                                                                                                                                                \\ \hline
\textbf{Input}                             & \multicolumn{1}{c|}{warehouseID}                                                                                                                                                                                           \\ \hline
\textbf{Output}                            & \begin{tabular}[c]{@{}l@{}}\{\\      "name": "ARMAZ",\\     "morada": "warehouse location",\\    "contacto": "warehouse contact"\\ \}\end{tabular} \\ \hline
\end{tabular}
\end{table}

\subsection{Transações}

\begin{table}[H]
	\centering
	\caption{Tabela do método get\_transaction.}
	\label{my-label}
	\begin{tabular}{|c|l|}
		\hline
		\textbf{Webservice ID}                     & \multicolumn{1}{c|}{get\_transaction}                                                                                                                                                                                        \\ \hline
		\multicolumn{1}{|l|}{\textbf{Description}} & Retorna a transação.
 \\ \hline
		\textbf{Core Views}                        & \multicolumn{1}{c|}{TRA\_PAGE}                                                                                                                                                                                                                        \\ \hline
		\textbf{Path}                              & \url{http://<domain>/api/transaction=<transactionID>}                                                                                                                                                          \\ \hline
		\textbf{Verb}                              & \multicolumn{1}{c|}{GET}                                                                                                                                                                                                \\ \hline
		\textbf{Input}                             & \multicolumn{1}{c|}{transactionID}                                                                                                                                                                                           \\ \hline
		\textbf{Output}                            & \begin{tabular}[c]{@{}l@{}}\{\\      "clienteID": "EFACSA",\\     "repvendaID": "VENDEDOR",\\    "produtos": [\{ "produtoID": 243,\\ "preco": 22.3,\\ "iva": 21,\\ "desconto": 0\},\\ ...],\\ "data": "2016-10-09 15:12:23" \}\end{tabular} \\ \hline
	\end{tabular}
\end{table}


\begin{table}[H]
	\centering
	\caption{Tabela do método post\_transaction.}
	\label{my-label}
	\begin{tabular}{|c|l|}
		\hline
		\textbf{Webservice ID}                     & \multicolumn{1}{c|}{post\_transaction}                                                                                                                                                                                        \\ \hline
		\multicolumn{1}{|l|}{\textbf{Description}} & Adiciona uma transação.                                                                                                                                                                                    \\ \hline
		\textbf{Core Views}                        & \multicolumn{1}{c|}{NEW\_TRA}                                                                                                                                                                                                                        \\ \hline
		\textbf{Path}                              & \url{http://<domain>/api/transaction}                                                                                                                                                          \\ \hline
		\textbf{Verb}                              & \multicolumn{1}{c|}{POST}                                                                                                                                                                                                \\ \hline
		\textbf{Input}                             & \multicolumn{1}{c|}{clienteID, repvendaID, listaProdutos}                                                                                                                                                                                           \\ \hline
		\textbf{Output}                            & \begin{tabular}[c]{@{}l@{}}\{\\  "resultado": "OK" \\ \}\end{tabular} \\ \hline
	\end{tabular}
\end{table}

\subsection{Artigos}

\begin{table}[H]
	\centering
	\caption{Tabela do método get\_article.}
	\label{my-label}
	\begin{tabular}{|c|l|}
		\hline
		\textbf{Webservice ID}                     & \multicolumn{1}{c|}{get\_article}                                                                                                                                                                                        \\ \hline
		\multicolumn{1}{|l|}{\textbf{Description}} & Retorna um artigo.                                                                                                                                                                                    \\ \hline
		\textbf{Core Views}                        & \multicolumn{1}{c|}{PRO\_PAGE}                                                                                                                                                                                                                        \\ \hline
		\textbf{Path}                              & \url{http://<domain>/api/article=<articleID>}                                                                                                                                                          \\ \hline
		\textbf{Verb}                              & \multicolumn{1}{c|}{GET}                                                                                                                                                                                                \\ \hline
		\textbf{Input}                             & \multicolumn{1}{c|}{articleID}                                                                                                                                                                                           \\ \hline
		\textbf{Output}                            & \begin{tabular}[c]{@{}l@{}}\{\\  "nome": "nome artigo",\\ "descricao": "descricao artigo",\\ "preco\_atual": 12.5,\\ "iva\_atual": 13 \\ \}\end{tabular} \\ \hline
	\end{tabular}
\end{table}


\section{Conclusão}
\justify\normalsize
Finda esta etapa, consideram-se atingidos os objetivos definidos para esta primeira fase: foi feita a descrição do projeto – uma aplicação web que oferecerá funcionalidades de uma solução SFA. Enumeram-se as funcionalidades a implementar – de entre as quais se desta-cam (exemplos) –, e identificaram-se os pontos de ligação com o ERP Primavera.

\bibliography{title_page_1}
\bibliographystyle{plain}

\end{titlepage}
\end{document}